\documentclass{article}
\usepackage[utf8]{inputenc}

\usepackage{hyperref}  % for hyperlinks
\hypersetup{
    colorlinks=true,
    linkcolor=blue,
    filecolor=magenta,
    urlcolor=cyan,
}

\usepackage{listings}  % for showing code

\title{Going From C++ to Python}
\author{Aaron Kaloti, VP of CS Tutoring Club in Winter 2019}
\date{March 5, 2019}

\begin{document}

\maketitle
\tableofcontents

\section{Introduction}

This is a guide to introductory Python 3 intended for those with a C++ background. It reviews the differences between C++ and Python 3 in introductory concepts such as conditional statements, strings, and functions. It is intended to make our UC Davis Computer Science tutors more comfortable tutoring the introductory Python courses -- ECS 32A, 32B, and 36A -- newly offered by UC Davis, as our school has shifted from teaching C/C++ at the introductory level to teaching Python.

\textbf{Want a short version of the major differences?} Besides syntactic differences, look at: // operator, ** operator, string immutability, range-based for loops, tuples, and dictionaries.

\section{(Quickly) Setting Up Python}
\textbf{Don't have Python 3?} Here are two solutions:
\begin{enumerate}
    \item Python 3 is already on the CSIF.
    \item Download Python3 from \href{https://www.python.org/downloads/}{here}. You may wish to download Python IDLE, to have a GUI (if you don't prefer using the terminal).
\end{enumerate}

\section{Running a Python Program}
\begin{itemize}
    \item Run a Python program like so, using the `python3` command. (If you're using IDLE, then do Run Module.) \textbf{Python is an interpreted language -- no compiler needed.}
    \begin{lstlisting}[language=bash]
    aaron123@ad3.ucdavis.edu@pc25:~$ cat hello-world.py
    def do_stuff():
        print("I did stuff")
    # Call the function we just defined.
    do_stuff()
    aaron123@ad3.ucdavis.edu@pc25:~$ python3 hello-world.py
    I did stuff
    aaron123@ad3.ucdavis.edu@pc25:~$
    \end{lstlisting}
    \begin{itemize}
        \item Note that we don't do `python hello-world.py`, as on the CSIF, `python` would run Python 2 instead of Python 3.
    \end{itemize}
    \item Use Interpreter Mode to try out things in Python.
    \begin{lstlisting}[language=bash]
    aaron123@ad3.ucdavis.edu@pc25:~$ python3
    Python 3.6.7 (default, Oct 22 2018, 11:32:17)
    [GCC 8.2.0] on linux
    Type "help", "copyright", "credits" or "license" for more information.
    >>> a = 3
    >>> a
    3
    >>> b = a
    >>> b
    3
    >>> quit()
    aaron123@ad3.ucdavis.edu@pc25:~$
    \end{lstlisting}
\end{itemize}

\section{General Differences Between C++ and Python}
In Python:
\begin{itemize}
    \item Lines don't end in semi-colons.
    \item Types of variables and function parameters aren't explicitly specified. If you do an operation/function on the wrong type of variable, you will get a runtime error.
    \item Python has automatic garbage collection and thus will "free" your no-longer-needed variables for you, so there's no malloc()/free() or new/delete.
    \item Comments are indicated by \# instead of //.
    \item "import" instead of "\#include".
\end{itemize}

\section{Numbers and Arithmetic}
\begin{itemize}
    \item +, -, *, and \% are the same.
    \item Unlike in C++, / doesn't truncate in Python when both operands are integers. You must use // to cause truncation.
    \begin{lstlisting}[language=Python]
    >>> 3 / 2
    1.5
    >>> 3 // 2
    1
    >>>
    \end{lstlisting}
    \item Use ** for exponentiation.
    \begin{lstlisting}[language=Python]
    >>> 5 ** 2  # 5 squared
    25
    \end{lstlisting}

\end{itemize}

\section{Standard Input/Output}
\begin{itemize}
    \item Use print() to print to standard output.
        \begin{itemize}
        \item For fans of C++'s printf():
        \begin{lstlisting}
        >>> print("{} says {}".format("Aaron","hi"))
        Aaron says hi
        \end{lstlisting}
        \end{itemize}
    \item Use input() for basic standard input.
    \begin{lstlisting}
    >>> name = input("Enter your name: ")
    Enter your name: Aaron
    >>> name
    'Aaron'
    >>>
    \end{lstlisting}
\end{itemize}

\section{Lists}
\begin{itemize}
    \item Lists in Python are arrays in C++, but you needn't do any special allocation stuff. The major list operations are demonstrated below.
    \item Define a list.
    \begin{lstlisting}
    >>> mylist = ['a','b','c','d','e']  # list of characters
    \end{lstlisting}
    \item Access element of a list.
    \begin{lstlisting}
    >>> mylist[0]  # Python uses zero-based indexing like C++
    'a'
    >>> mylist[4]
    'e'
    \end{lstlisting}
    \item Access element of a list, starting from the back.
    \begin{lstlisting}
    >>> mylist[-1]  # get first element from back
    'e'
    >>> mylist[-2]
    'd'
    \end{lstlisting}
    \item Get length of a list.
    \begin{lstlisting}
    >>> len(mylist)  # get length of list
    5
    \end{lstlisting}
    \item Splice a sublist from the list.
    \begin{lstlisting}
    >>> mylist[0:2]  # splice from index 0 to before index 2
    ['a', 'b']
    >>> mylist[1:4]  # splice from index 1 to before index 4
    ['b', 'c', 'd']
    >>> mylist[2:]  # splice from index 2 to end
    ['c', 'd', 'e']
    >>> mylist[-2:]  # splice from second-to-last element onwards
    ['d', 'e']
    \end{lstlisting}
    \item Concatenate two lists.
    \begin{lstlisting}
    >>> ['t','u','v'] + ['w','x']  # concatenation
    ['t', 'u', 'v', 'w', 'x']
    \end{lstlisting}
    \item Modify a specific list element.
    \begin{lstlisting}
    >>> mylist[2] = 'x'  # modification: replace 'c' with 'x'
    >>> mylist
    ['a', 'b', 'x', 'd', 'e']
    \end{lstlisting}
    \item Append an element to a list.
    \begin{lstlisting}
    >>> mylist.append('z')  # append 'z' to back of list
    >>> mylist
    ['a', 'b', 'x', 'd', 'e', 'z']
    \end{lstlisting}
    \item Delete an element from the list.
    \begin{lstlisting}
    >>> del mylist[1]  # delete 'b' from list
    >>> mylist
    ['a', 'x', 'd', 'e', 'z']
    \end{lstlisting}
    \item Get the type of this list.
    \begin{lstlisting}
    >>> type(mylist)
    <class 'list'>
    \end{lstlisting}
    \item You may hear a Python list be called a "sequence data type", since it supports indexing.
\end{itemize}

\section{Strings}
\begin{itemize}
    \item Strings in Python are strings in C++, but Python has no characters (a character is a string of length 1).
    \item No difference between single quote and double quote.
    \item Ignoring modification operations, strings and lists have the same operations.
    \item Define a string.
    \begin{lstlisting}
    >>> mystr = "aaron kaloti"
    \end{lstlisting}
    \item Access element of a string.
    \begin{lstlisting}
    >>> mystr[0]  # again, zero-based indexing
    'a'
    >>> mystr[-2]  # second-to-last character
    't'
    \end{lstlisting}
    \item Get length of a string.
    \begin{lstlisting}
    >>> len(mystr)
    12
    \end{lstlisting}
    \item Splice a substring from the string.
    \begin{lstlisting}
    >>> mystr[:5]  # splice to get my first name
    'aaron'
    >>> mystr[6:]  # splice to get my last name
    'kaloti'
    \end{lstlisting}
    \item Concatenate two strings.
    \begin{lstlisting}
    >>> "concat" + "enation"
    'concatenation'
    \end{lstlisting}
    \item Get the type of this string.
    \begin{lstlisting}
    >>> type(mystr)
    <class 'str'>
    \end{lstlisting}
    \item IMPORTANT: In Python, we call strings "immutable". This means that, unlike with a list, \textbf{you can't modify an individual element in a string}. You must instead use concatenation to create a new string.
    \begin{lstlisting}
    >>> mystr
    'aaron kaloti'
    >>> mystr[1] = 'd'
    >>> mystr[3] = 'i'
    Traceback (most recent call last):
      File "<stdin>", line 1, in <module>
    TypeError: 'str' object does not support item assignment
    >>> mystr = mystr[:3] + 'i' + mystr[4:]
    >>> mystr
    'aarin kaloti'
    \end{lstlisting}
    \item You may hear a Python string be called a "sequence data type", since it supports indexing.
\end{itemize}

\section{Conditional Statements}
\begin{itemize}
    \item If/else statements are the same, besides syntactic differences (no parentheses around the condition, condition ends with a colon, indentation indicates the body of the if/else, and we use "elif" instead of "else if"):
    \begin{lstlisting}
    >>> x = 8
    >>> if x < 0:
    ...     print('x is negative')
    ... elif x == 0:
    ...     print('x is zero')
    ... else:
    ...     print('x is positive')
    ...
    x is positive
    >>>
    \end{lstlisting}
\end{itemize}

\section{Iteration}
\begin{itemize}
    \item While loops are the same, besides minor syntactic differences:
    \begin{lstlisting}
    >>> mylist = ['dog','cat','mouse']
    >>> i = 0
    >>> while i < len(mylist):
    ...     print(mylist[i])
    ...     i += 1
    ...
    dog
    cat
    mouse
    >>>
    \end{lstlisting}
    \item Range-based for loop: for loop to iterate across a range of values (here, the variable \textbf{i} needn't be initialized prior):
    \begin{lstlisting}
    >>> for i in range(2,6):  # last value (6) isn't included
    ...     print(i)
    ...
    2
    3
    4
    5
    >>> for i in range(3):  # range starts at 0 if only give one value
    ...     print(i)
    ...
    0
    1
    2
    >>>
    \end{lstlisting}
    \item For loop to iterate across the values in a list:
    \begin{lstlisting}
    >>> people = ['Aaron','Aakash','Matthew']
    >>> for person in people:
    ...     print(person)
    ...
    Aaron
    Aakash
    Matthew
    >>>
    \end{lstlisting}
    \begin{itemize}
        \item Note that when using this syntax, we can't change the values in the list.
        \begin{lstlisting}
        >>> people = ['Aaron','Aakash','Matthew']
        >>> for person in people:
        ...     person = "Alex"
        ...
        >>> people  # note that the list is unaffected
        ['Aaron', 'Aakash', 'Matthew']
        >>>
        \end{lstlisting}
    \end{itemize}
    \item \textbf{break} and \textbf{continue} work the same.
\end{itemize}

\section{Functions}
\begin{itemize}
    \item Types of function parameters aren't specified.
    \item A return type can't be specified in Python, so a function can return different types of values (or no value at all).
    \item Here is an example to illustrate syntactic differences:
    \begin{lstlisting}
    >>> def isEven(val):
    ...     if val % 2 == 0:
    ...             return True
    ...     else:
    ...             return False
    ...
    >>> isEven(3)
    False
    >>> isEven(4)
    True
    >>>
    \end{lstlisting}
    \item Default argument values:
    \begin{lstlisting}
    >>> def returnInput(val=8):
    ...     return val
    ...
    >>> returnInput(3)
    3
    >>> returnInput()  # use default argument
    8
    >>>
    \end{lstlisting}
\end{itemize}

\section{Tuples}
\begin{itemize}
    \item A tuple is basically a list, except each individual element is immutable.
    \item Define a tuple.
    \begin{lstlisting}
    >>> t = (8,5.3,'blah')
    >>> t
    (8, 5.3, 'blah')
    >>> t = 8,5.3,'blah'  # you might see it without the parentheses
    >>> t
    (8, 5.3, 'blah')
    >>>
    \end{lstlisting}
    \item Access element in a tuple.
    \begin{lstlisting}
    >>> t[1]
    5.3
    >>> t[-1]
    'blah'
    \end{lstlisting}
    \item Get length of a tuple.
    \begin{lstlisting}
    >>> len(t)
    3
    \end{lstlisting}
    \item Splice from a tuple.
    \begin{lstlisting}
    >>> t[1:]
    (5.3, 'blah')
    \end{lstlisting}
    \item Concatenate two tuples.
    \begin{lstlisting}
    >>> ('part',1) + ('part',2)
    ('part', 1, 'part', 2)
    \end{lstlisting}
    \item Get the type of this tuple.
    \begin{lstlisting}
    >>> type(t)
    <class 'tuple'>
    \end{lstlisting}
    \item IMPORTANT: In Python, we call tuples "immutable", so -- as with strings -- \textbf{you can't modify an individual element in a tuple}. You must instead create a new tuple with concatenation.
    \begin{lstlisting}
    >>> t[2] = 'nah'
    Traceback (most recent call last):
      File "<stdin>", line 1, in <module>
    TypeError: 'tuple' object does not support item assignment
    >>>
    \end{lstlisting}
    \item You may hear a Python tuple be called a "sequence data type", since it supports indexing.
\end{itemize}

\section{Sets (might not be covered in ECS 32A, 32B, or 36A)}
\begin{itemize}
    \item A set is a collection/container that \textbf{ignores duplicates}.
    \item Define a set with curly braces. Note the removal of the "Billy" duplicate.
    \begin{lstlisting}
    >>> people = {"Bob", "Billy", "Blake", "Billy"}
    >>> people
    {'Bob', 'Billy', 'Blake'}
    \end{lstlisting}
    \item A set is \textit{not} a sequence data type and thus \textbf{cannot be indexed}.
    \begin{lstlisting}
    >>> people[2]
    Traceback (most recent call last):
      File "<stdin>", line 1, in <module>
    TypeError: 'set' object does not support indexing
    \end{lstlisting}
    \item However, membership testing can be performed.
    \begin{lstlisting}
    >>> 'Bob' in people
    True
    >>> 'Ryan' in people
    False
    \end{lstlisting}
    \item \textbf{For those curious about standard library implementations}, note that \textbf{C++'s std::set is not akin to Python's set}. C++'s std::set sorts its elements and is implemented by a self-balancing binary search tree (a red-black tree, I believe), but Python's set does \textit{not} sort its elements and is implemented by a hash table. Thus, \textbf{Python's set is akin to C++'s std::unordered\_set}.
\end{itemize}

\section{Dictionaries}
\begin{itemize}
    \item Dictionaries are indexed by key and return a value.
    \begin{lstlisting}
    >>> d = {'apple': 33, 'teehee': 21}
    >>> d
    {'apple': 33, 'teehee': 21}
    >>> d['apple']
    33
    \end{lstlisting}
    \item Append to a dictionary like so (note that keys and values needn't be consistent types).
    \begin{lstlisting}
    >>> d['blah'] = 'hi'
    >>> d[4] = 8
    >>> d
    {'apple': 33, 'teehee': 21, 'blah': 'hi', 4: 8}
    >>>
    \end{lstlisting}
    \item Get length:
    \begin{lstlisting}
    >>> len(d)
    4
    \end{lstlisting}
    \item Delete an element:
    \begin{lstlisting}
    >>> del d['teehee']
    >>> d
    {'apple': 33, 'blah': 'hi', 4: 8}
    \end{lstlisting}
    \item One way to iterate through a dictionary:
    \begin{lstlisting}
    >>> for k in d:
    ...     print(k,d[k])
    ...
    apple 33
    blah hi
    4 8
    \end{lstlisting}
    \item Another way to iterate through a dictionary:
    \begin{lstlisting}
    >>> for k,v in d.items():
    ...     print(k,v)
    ...
    apple 33
    blah hi
    4 8
    \end{lstlisting}
\end{itemize}

\section{File Input/Output}
\begin{itemize}
    \item Opening a file, reading all of it, and closing it:
    \begin{lstlisting}
    aaron123@ad3.ucdavis.edu@pc25:~$ cat poem.txt
    There once was a man from Peru
    Who dreamed he was eating his shoe.
    He woke up with a fright
    In the middle of the night
    To find that his dream had come true.
    aaron123@ad3.ucdavis.edu@pc25:~$ python3
    >>> fr = open('poem.txt', 'r')
    >>> fr.read()  # read entire file
    'There once was a man from Peru\nWho dreamed he was eating his shoe.\nHe woke up with a fright\nIn the middle of the night\nTo find that his dream had come true.\n'
    >>> fr.readline()  # nothing left to read
    ''
    >>> fr.close()
    \end{lstlisting}
    \item Can read a file line by line.
    \begin{lstlisting}
    >>> fr = open('poem.txt', 'r')
    >>> fr.readline()
    'There once was a man from Peru\n'
    >>> fr.readline()
    'Who dreamed he was eating his shoe.\n'
    \end{lstlisting}
    \item Writing to a file:
    \begin{lstlisting}
    >>> fw = open('poem.txt', 'w')  # clears file's contents
    >>> fw.write("nah\n")  # returns number of characters written
    4
    >>> fw.close()
    >>> quit()
    aaron123@ad3.ucdavis.edu@pc25:~$ cat poem.txt
    nah
    aaron123@ad3.ucdavis.edu@pc25:~$
    \end{lstlisting}
    \item Use 'r+', NOT 'rw', to open a file for reading and writing.
\end{itemize}

\section{Command-line arguments}
\begin{itemize}
    \item Minor syntactic differences, demonstrated below. Python has no argc; use len(sys.argv) instead.
    \begin{lstlisting}
    aaron123@ad3.ucdavis.edu@pc25:~$ cat print-args.py
    import sys
    print("Arguments: ", sys.argv)
    print("Number of arguments: ", len(sys.argv))
    aaron123@ad3.ucdavis.edu@pc25:~$ python3 print-args.py aaron kaloti
    Arguments:  ['print-args.py', 'aaron', 'kaloti']
    Number of arguments:  3
    aaron123@ad3.ucdavis.edu@pc25:~$ python3 print-args.py
    Arguments:  ['print-args.py']
    Number of arguments:  1
    aaron123@ad3.ucdavis.edu@pc25:~$
    \end{lstlisting}
\end{itemize}

\section{Exceptions}
\begin{itemize}
    \item Minor syntactic differences, demonstrated below (example's \href{https://docs.python.org/3/tutorial/errors.html}{source}).
    \begin{lstlisting}
    >>> while True:
    ...     try:
    ...             x = int(input("Please enter a number: "))
    ...             break
    ...     except ValueError:
    ...             print("Oops!  That was no valid number.  Try again...")
    ...
    Please enter a number: abc
    Oops!  That was no valid number.  Try again...
    Please enter a number: 5
    >>>
    \end{lstlisting}
\end{itemize}

\section{Basic User-Defined Classes (for small part of ECS 32B)}
\begin{itemize}
    \item NOTE: User-defined classes shouldn't come up in ECS 32A or 36A, but they do come up briefly in Kurt Eiselt's ECS 32B. Kurt doesn't cover inheritance, but if a future ECS 32B instructor does, I'll add content about inheritance to this guide.
    \item In Python, we use "self" instead of "this". Python's "self" is always required, unlike C++'s "this", which is implicit in a lot of cases.
    \item Unlike C++, Python doesn't truly support private members (just ways to make "private" members harder to access, but I won't get into that here.)
    \item Here is a sample user-defined class:
    \lstinputlisting[language=Python]{sample_class.py}
    \item Let's play with this class:
    \begin{lstlisting}
    aaron123@ad3.ucdavis.edu@pc25:~$ python3
    >>> from sample_class import Person
    >>> a = Person("Bob", 34)
    >>> a.get_age()
    34
    >>> a.set_age(88)
    >>> a.age  # no privacy
    88
    >>> a.name  # no privacy
    'Bob'
    \end{lstlisting}
\end{itemize}

\section{Other Brief Tutorials}
\begin{itemize}
    \item W3Schools' new Python tutorial, provided \href{https://www.w3schools.com/python/default.asp}{here}.
    \item Tutorial in the Python 3 documentation, although I find it too extensive (as it was intended to be), provided \href{https://docs.python.org/3/tutorial/}{here}.
\end{itemize}

\end{document}
