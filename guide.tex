\documentclass{article}
\usepackage[utf8]{inputenc}

\usepackage{hyperref}  % for hyperlinks
\hypersetup{
    colorlinks=true,
    linkcolor=blue,
    filecolor=magenta,      
    urlcolor=cyan,
}

\usepackage{listings}  % for showing code
 
\title{Going From C++ to Python}
\author{Aaron Kaloti, VP of CS Tutoring Club in Winter 2019}
\date{March 5, 2019}

\begin{document}

\maketitle

\section{Introduction}

This is a guide to introductory Python 3 intended for those with a C++ background. It reviews the differences between C++ and Python 3 in introductory concepts such as conditional statements, strings, and functions. It is intended to make our UC Davis Computer Science tutors more comfortable teaching the introductory Python courses -- ECS 32A, 32B, and 36A -- newly offered by UC Davis, as our school shifts from teaching C/C++ at the introductory level to teaching Python.

\section{(Quickly) Setting Up Python}
\textbf{Don't have Python 3?} Here are two solutions:
\begin{enumerate}
    \item Python 3 is already on the CSIF.
    \item Download Python3 from \href{https://www.python.org/downloads/}{here}. You may wish to download Python IDLE, to have a GUI (if you don't prefer using the terminal). 
\end{enumerate}

\section{Running a Python Program}
\begin{itemize}
    \item Run a Python program like so, using the `python3` command. (If you're using IDLE, then do Run Module.) \textbf{Python is an interpreted language -- no compiler needed.}
    \begin{lstlisting}[language=bash]
    aaron123@ad3.ucdavis.edu@pc25:~$ cat hello-world.py 
        def do_stuff():
            print("I did stuff")
        
        # Call the function we just defined.
        do_stuff()
    aaron123@ad3.ucdavis.edu@pc25:~$ python3 hello-world.py 
    I did stuff
    aaron123@ad3.ucdavis.edu@pc25:~$ 
    \end{lstlisting}
    \begin{itemize}
        \item Note that we don't do `python hello-world.py`, as on the CSIF, `python` would run Python 2 instead of Python 3.
    \end{itemize}
    \item Use Interpreter Mode to try out things in Python.
    \begin{lstlisting}[language=bash]
    aaron123@ad3.ucdavis.edu@pc25:~$ python3
    Python 3.6.7 (default, Oct 22 2018, 11:32:17) 
    [GCC 8.2.0] on linux
    Type "help", "copyright", "credits" or "license" for more information.
    >>> a = 3
    >>> a
    3
    >>> b = a
    >>> b
    3
    >>> quit()
    aaron123@ad3.ucdavis.edu@pc25:~$ 
    \end{lstlisting}
\end{itemize}

\section{General Differences}
\begin{itemize}
    \item Lines don't end in semi-colons.
    \item Types of variables and function parameters aren't explicitly specified.
    \item Python has automatic garbage collection and thus will "free" your no-longer-needed variables for you, so there's no malloc()/free() or new/delete.
    \item Comments are indicated by # instead of //.
\end{itemize}

\section{Numbers and Arithmetic}
\begin{itemize}
    \item +, -, *, and \% are the same.
    \item Unlike in C++, / doesn't truncate in Python when both operands are integers. You must use // to cause truncation.
    \begin{lstlisting}[language=Python]
    >>> 3 / 2
    1.5
    >>> 3 // 2
    1
    >>> 
    \end{lstlisting}
    \item Use ** for exponentiation.
    \begin{lstlisting}[language=Python]
    >>> 5 ** 2  # 5 squared
    25
    \end{lstlisting}

\end{itemize}

\section{Standard Input/Output}
\begin{itemize}
    \item Use print() to print to standard output.
        \begin{itemize}
        \item For fans of C++'s printf():
        \begin{lstlisting}
        >>> print("{} says {}".format("Aaron","hi"))
        Aaron says hi
        \end{lstlisting}
        \end{itemize}
    \item Use input() for basic standard input.
    \begin{lstlisting}
    >>> name = input("Enter your name: ")
    Enter your name: Aaron
    >>> name
    'Aaron'
    >>> 
    \end{lstlisting}
\end{itemize}

\section{Lists}
\begin{itemize}
    \item Lists in Python are arrays in C++, but you needn't do any special allocation stuff. The major list operations (access, length, splicing, concatenation, modification, append, delete) are demonstrated:
    \item Define a list.
    \begin{lstlisting}
    >>> mylist = ['a','b','c','d','e']  # list of characters
    \end{lstlisting}
    \item Access element of a list.
    \begin{lstlisting}
    >>> mylist[0]  # Python uses zero-based indexing like C++
    'a'
    >>> mylist[4]
    'e'
    \end{lstlisting}
    \item Access element of a list, starting from the back.
    \begin{lstlisting}
    >>> mylist[-1]  # get first element from back
    'e'
    >>> mylist[-2]
    'd'
    \end{lstlisting}
    \item Get length of a list.
    \begin{lstlisting}
    >>> len(mylist)  # get length of list
    5
    \end{lstlisting}
    \item Splice a sublist from the list.
    \begin{lstlisting}
    >>> mylist[0:2]  # splice from index 0 to before index 2
    ['a', 'b']
    >>> mylist[1:4]  # splice from index 1 to before index 4
    ['b', 'c', 'd']
    >>> mylist[2:]  # splice from index 2 to end
    ['c', 'd', 'e']
    >>> mylist[-2:]  # splice from second-to-last element onwards
    ['d', 'e']
    \end{lstlisting}
    \item Concatenate two lists.
    \begin{lstlisting}
    >>> ['t','u','v'] + ['w','x']  # concatenation
    ['t', 'u', 'v', 'w', 'x']
    \end{lstlisting}
    \item Modify a specific list element.
    \begin{lstlisting}
    >>> mylist[2] = 'x'  # modification: replace 'c' with 'x'
    >>> mylist
    ['a', 'b', 'x', 'd', 'e']
    \end{lstlisting}
    \item Append an element to a list.
    \begin{lstlisting}
    >>> mylist.append('z')  # append 'z' to back of list
    >>> mylist
    ['a', 'b', 'x', 'd', 'e', 'z']
    \end{lstlisting}
    \item Delete an element from the list.
    \begin{lstlisting}
    >>> del mylist[1]  # delete 'b' from list
    >>> mylist
    ['a', 'x', 'd', 'e', 'z']
    \end{lstlisting}
    \item Get the type of this list.
    \begin{lstlisting}
    >>> type(mylist)
    <class 'list'>
    \end{lstlisting}
\end{itemize}

\section{Strings}
\begin{itemize}
    \item Strings in Python are strings in C++, but Python has no characters (a character is a string of length 1).
    \item No difference between single quote and double quote.
    \item Ignoring modification operations, strings and lists have the same operations.
    \item Define a string.
    \begin{lstlisting}
    >>> mystr = "aaron kaloti"
    \end{lstlisting}
    \item Access element of a string.
    \begin{lstlisting}
    >>> mystr[0]  # again, zero-based indexing
    'a'
    >>> mystr[-2]  # second-to-last character
    't'
    \end{lstlisting}
    \item Get length of a string.
    \begin{lstlisting}
    >>> len(mystr)
    12
    \end{lstlisting}
    \item Splice a substring from the string.
    \begin{lstlisting}
    >>> mystr[:5]  # splice to get my first name
    'aaron'
    >>> mystr[6:]  # splice to get my last name
    'kaloti'
    \end{lstlisting}
    \item Concatenate two strings.
    \begin{lstlisting}
    >>> "concat" + "enation"
    'concatenation'
    \end{lstlisting}
    \item Get the type of this string.
    \begin{lstlisting}
    >>> type(mystr)
    <class 'str'>
    \end{lstlisting}
    \item IMPORTANT: In Python, we call strings "immutable". This means that, unlike with a list, \textbf{you can't modify an individual element in a string}. If you want to change a string, you must use concatenation (to create a new string).
    \begin{lstlisting}
    >>> mystr
    'aaron kaloti'
    >>> mystr[1] = 'd'
    >>> mystr[3] = 'i' 
    Traceback (most recent call last):
      File "<stdin>", line 1, in <module>
    TypeError: 'str' object does not support item assignment
    >>> mystr = mystr[:3] + 'i' + mystr[4:]
    >>> mystr
    'aarin kaloti'
    \end{lstlisting}
\end{itemize}

\section{Conditional Statements}
\begin{itemize}
    \item If/else statements are the same, besides syntactic differences (no parentheses around the condition, condition ends with a colon, indentation indicates the body of the if/else, and we use "elif" instead of "else if"):
    \begin{lstlisting}
    >>> x = 8
    >>> if x < 0:
    ...     print('x is negative')
    ... elif x == 0:
    ...     print('x is zero')
    ... else:
    ...     print('x is positive')
    ... 
    x is positive
    >>> 
    \end{lstlisting}
\end{itemize}

\section{Iteration}
\begin{itemize}
    \item While loops are the same, besides minor syntactic differences:
    \begin{lstlisting}
    >>> mylist = ['dog','cat','mouse']
    >>> i = 0
    >>> while i < len(mylist):
    ...     print(mylist[i])
    ...     i += 1
    ... 
    dog
    cat
    mouse
    >>> 
    \end{lstlisting}
    \item For loop to iterate across a range of values (here, the variable \textbf{i} needn't be initialized prior):
    \begin{lstlisting}
    >>> for i in range(2,6):  # last value (6) isn't included
    ...     print(i)
    ... 
    2
    3
    4
    5
    >>> for i in range(3):  # range starts at 0 if only give one value
    ...     print(i)
    ... 
    0
    1
    2
    >>> 
    \end{lstlisting}
    \item For loop to iterate across the values in a list:
    \begin{lstlisting}
    >>> people = ['Aaron','Aakash','Matthew']
    >>> for person in people:
    ...     print(person)
    ... 
    Aaron
    Aakash
    Matthew
    >>>
    \end{lstlisting}
    \begin{itemize}
        \item Note that when using this syntax, we can't change the values in the list.
        \begin{lstlisting}
        >>> people = ['Aaron','Aakash','Matthew']
        >>> for person in people:
        ...     person = "Alex"
        ... 
        >>> people  # note that the list is unaffected
        ['Aaron', 'Aakash', 'Matthew']
        >>> 
        \end{lstlisting}
    \end{itemize}
    \item \textbf{break} and \textbf{continue} work the same.
\end{itemize}

\section{Functions}
\begin{itemize}
    \item Types of function parameters aren't specified.
    \item No return type is specified, so a function can return different types of values (or in some cases, no value at all).
    \item Here is an example to illustrate syntactic differences:
    \begin{lstlisting}
    >>> def isEven(val):
    ...     if val % 2 == 0:
    ...             return True
    ...     else:
    ...             return False
    ... 
    >>> isEven(3)
    False
    >>> isEven(4)
    True
    >>> 
    \end{lstlisting}
    \item Default argument values:
    \begin{lstlisting}
    >>> def returnInput(val=8):
    ...     return val
    ... 
    >>> returnInput(3)
    3
    >>> returnInput()  # use default argument
    8
    >>> 
    \end{lstlisting}
\end{itemize}

\section{Tuples}

\section{Dictionaries}

\section{File Input/Output}
\begin{itemize}
\end{itemize}

\section{Command-line arguments}

\section{Exceptions}


\section{Classes}
\begin{itemize}
    \item NOTE: User-defined classes shouldn't come up in ECS 32A or 36A, but they do come up briefly in Kurt Eiselt's ECS 32B. (He doesn't cover inheritance, but if a future ECS 32B instructor does, I'll update this guide.)
\end{itemize}

\end{document}

